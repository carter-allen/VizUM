\documentclass[12pt]{article}
\usepackage{parskip}
\usepackage[margin=.75in]{geometry}
\usepackage{hyperref}
\hypersetup{
    colorlinks=true,
    linkcolor=blue,
    filecolor=magenta,      
    urlcolor=cyan,
}
\linespread{1}

\title{Idea Checkpoint}
\author{Carter Allen}

\begin{document}
\maketitle
\raggedright

For this project I plan to look at the distribution of public parks in Miami-Dade county using data from the county's open data portal found \href{https://opendata.miamidade.gov/Infrastructure/Parks-Facities/wjhr-nx6u}{here}. My motivation for working with public parks data in for this project is my belief that public spaces are a cornerstone of a healthy society, as they promote both self reflection and community involvement. I have studied the importance of the natural parks system in the United States, and \href{https://sufficientlyminimal.netlify.com/2018/01/28/congaree-big-trees/}{worked with} national parks data for an analysis before, so choosing to focus on public spaces in Miami-Dade for this project was a natural extension of this interest. My progress thus far has been to make an interactive \href{https://carter-allen.shinyapps.io/Miami_Parks/}{map} of the public parks and spaces in Miami-Dade county. With this map, a user can vizualize the distribution of parks in the area. Each point on the map is sized in proportion to the park's total acreage, and the user can filter the parks shown on the map according to the activities offered by the park (i.e. baseball, dog park, beach, etc...). Moving forward, I hope to incorporate more dimensions to this work, such as looking at possible relations between an areas socioeconomic status and the availability of public spaces to its residents. Unfortunately, it is often only the wealthy communities that enjoy access to high quality public spaces in cities around the country. I look forward to hearing feedback on these ideas and discuss suggestions for future directions.

\textit{Note: the materials for this project are accessible \href{https://carter-allen.github.io/VizUM/}{here}}.

\end{document}